\documentclass[12pt]{article}

\usepackage{pdfsync}
\usepackage{amsmath}
\usepackage{graphicx}

\graphicspath{{./figures/}}

%%%%%%%%%%%%%%%%%%%%%%%%%%%%%%%%%%%%%%%%%%%%%%%%%%%%%%%%%%%%
\renewcommand{\d}{\mathrm{d}}
\newcommand{\vv}[1]{\boldsymbol{#1}}
\newcommand{\cB}{B}
\newcommand{\cA}{A}
\newcommand{\cC}{C}
\usepackage{textcomp}
\usepackage[T1]{fontenc}


%%%%%%%%%%%%%%%%%%%%%%%%%%%%%%%%%%%%%%%%%%%%%%%%%%%%%%%%%%%%


\begin{document}
\section{Derivation of the Equations of Motion
}\label{Derivation of the Equations of Motion
}
\subsection{Some highlights of Hamilton's Principle}
\subsection{Derivation of the Equations of Motion}
Under the action of forces, the elastic rod-like filament suffers deformations, which we assume to be planar. Whenever these deformations occur, it will be a contact force \textbf{n}(s,t) exerted by the material of [s,1] on that of [0,s). \textbf{n}(s,t) is assumed to lie to the (\textbf{e_1,e_2})-plane, so that it has the form
\[  \mathbf{n}(s,t)= N(s,t) \mathbf{a}(s,t) + H(s,t)\mathbf{b} (s,t).
\]
Let \mathbf{m}(s,t) be the contact couple exerted by the material of [s,1] on tha of [0,s). We assume that \mathbf{m} is perpendicular to the (\mathbf{e_1,e_2})-plane so that it has the form
\[ 
\mathbf{m}(s,t) = M(s,t)\mathbf{e}_3.
\]

We assume that the material of the rod is viscoelastic causing internal energy dissipation and having constitutive equations of the form
\[  N (s,t)=  \hat{N}(\nu (s,t),\eta (s,t)) + A \nu_t (s,t)
\]
\[  H (s,t)=  \hat{H}(\nu (s,t),\eta (s,t)) + B \eta_t (s,t)
\]
\[  M (s,t)=  \hat{M}(\nu (s,t),\eta (s,t), \mu(s,t)) + C \mu_t (s,t),
\]
where \^{N},\^{H},\^{M}are given functions on (0, \infty ) \textmultiply \textcolonmonetary{R} \textmultiply \textcolonmonetary{R}
We model this by introducing the elastic potential density  as
\[ \mathcal{U}(\mathbf{y}) = \frac{\cA}{2} (\nu - \nu_0)^2
+\frac{\cC}{2} (\eta - \eta_0 )^2
+\frac{\cB}{2} (\mu -\alpha)^2 
\]
and the internal energy dissipation density as

\]
\end{document}
  
  
  
  
  
  
  
  
  
  
  
  
  
  
  
  