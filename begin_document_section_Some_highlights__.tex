\begin{document}
\section{Some highlights of Hamilton's Principle}
Consider a system of particles in  whose position, or $
\emph{configuration}, can be described by a set of independent generalized coordinates $\mathbf{q}_{k}$, k = 1, 2, . . . ,K. Let t_1 and t_2 be fixed times, with t_1 < t_2, and suppose that $\mathbf{q}$(t_1) and $\mathbf{q}$(t_2) are prescribed.\\
An \emph{admissible} motion of the system will be defined to be a set of functions $\mathbf{q}$_{k}(t), k = 1, 2, . . .,K, which satisfy the prescribed values at t_1 and t_2 and are C^2 on [t_1, t_2].\\

To translate in mathematical terms the fact that there might be different configurations, we introduce a \emph{perturbation} $\delta\mathbf{q}$_k, which is an arbitrary C^2 function on [t_1,t_2] subject to the requirements that $\delta\mathbf{q}$_k(t_1)= 0 and $\delta\mathbf{q}$_k(t_2)=0.\\
It will be convenient to introduce the notation
\[  \delta(\cdot)=\left( \frac{\partial}{\partial\epsilon}(\cdot)^{\ast}\right) \]
The symbol $\delta(\cdot)$ is called the variation of the expression $(\cdot)$.





\end{document}
  
  
  
  