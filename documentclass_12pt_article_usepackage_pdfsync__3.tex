\documentclass[12pt]{article}

\usepackage{pdfsync}
\usepackage{amsmath}
\usepackage{graphicx}
\usepackage{textcomp}
\usepackage[T1]{fontenc}

\graphicspath{{./figures/}}

%%%%%%%%%%%%%%%%%%%%%%%%%%%%%%%%%%%%%%%%%%%%%%%%%%%%%%%%%%%%
\renewcommand{\d}{\mathrm{d}}
\newcommand{\vv}[1]{\boldsymbol{#1}}
\newcommand{\cB}{B}
\newcommand{\cA}{A}
\newcommand{\cC}{C}



%%%%%%%%%%%%%%%%%%%%%%%%%%%%%%%%%%%%%%%%%%%%%%%%%%%%%%%%%%%%


\begin{document}
\begin{center}
Abstract
\end{center}
Beck's column is a classical problem of structural mechanics, where a cantilever column is clamped on one end, and it is subject to a compressive follower force on the other end. Beck's column is one of the simplest mechanical models which presents a Hopf's bifurcation when the follower load exceeds a critical value. 

In the classical Beck's column problem, a dynamic instability leading to a Hopf bifurcation is observed only when one includes inertial terms in the formulation, together with viscoelastic terms in the elastic response of the column. In such cases, the viscoelastic terms are the ones responsible for the destabilization.  

We study a variation of Beck's column, by considering the column immersed in a viscous fluid and we investigate the minimal conditions which trigger a Hopf bifurcation. In this generalized case, a Hopf bifurcation is observed also when neglecting both inertial and viscoelastic terms. Our observations indicate that viscous effects remain essential to trigger Hopf bifurcations, but they need not be of viscoelastic type. The presence of a viscous fluid surrounding the column allows instabilities to occur also when neglecting both inertial and viscoelastic effects, contrarily to what is commonly believed in the literature.

Both the linearized and the fully nonlinear problems are analyzed, and a discretization by iso-geometric Finite Element Methods is presented, together with several numerical experiments. 


\section{Introduction}
(Work in progress)
\\
The column, usually referred to as Beek's column, is clamped at one end and is subject to a follower force at the  free end.
Such a force has the defining property that its magnitude is fixed but its direction must remain normal to the cross-section of the column at the end where this force is applied.
\\\\
In Section 1, we model the column as an extensional shearable rod and formulate our problem by using the simple Cosserat model. 
The method of Cosserat dynamics for elastic structures is employed since it can accommodate to a good approximation the nonlinear behavior of complex elastic structures composed of materials with different constitutive properties, variable geometry and damping characteristics. With arbitrary boundary conditions, the Cosserat theory is used to formulate a set of governing partial differential equations of motion in terms of the displacements and angular variables, describing the planar, nonlinear dynamics of an extensional rod. Bending about two principal axes, extension, shear and torsion are considered, and care is taken into account for all the nonlinear terms in the resulting equations. 
\\\\
In Section 2, we explore the planar, nonlinear dynamics of the rod by explicitly deriving, through a generalization of Hamilton's Variational Principle, the equations of motion. We take into account the kinetic energy, the stored potential energy and the virtual work done by the dissipative forces due to the viscoelasticity of the material. Moreover we use resistive force theory to model the force applied by fluid on the filament rod. We use an approximation for writing the force applied by fluid after observing that using the exact form of the force would yield to an extremely complicated calculus problem. We linearize and use a discretization in week form by using FEM.
\\\\
After that we repeat the same studies by imposing the constraints that the rod is inextensible and unshearable.
We observe that the dynamic equations of motion have the same structure in the two different models. After adding to the equations of motion the three constitutive equations that model the viscoelasticity of the rod in the first model, and the equations of the constrains in second model, we notice that these two different treatments yield to two different final matrix formulations.
\\\\
In Section 3, after some theoretical results on Hopf bifurcation, our main goal is to observe the same bifurcation in absence of mass, and in absence of viscosity.
\\\\
Several numerical experiments follow.


\section{Background on the special Cosserat model}
(Work in progress)\\
We briefly summarize the basic aspects of the special Cosserat theory of rods using a notation adopted from Antman. Generally, a rod is a fiber–like elastic body, i.e. it is possible to specify a family of cross–sections which have small proportions compared to the length of the rod. This suggests a mathematical model of a rod in terms of a spacecurve corresponding to its centerline and a director frame which defines the orientation of the local cross–section plane.
Throughout this work, we denote three–dimensional Euclidean space by ($\mathbb{E}$^3, ⟨ · , · ⟩) and choose a fixed right–handed orthonormal triple ($\mathbf{e}$_1, $\mathbf{e}$_2, $\mathbf{e}$_3) of basis vectors.\\\\
$\mathbf{Configuration.}$ Let a thin 2-dimensional body occupy a rectangular region
\[  \mathcal{R}= \left \{
x\mathbf{e}_1 + y\mathbf{e}_2: x\in[0,1] ,y\in[-h,h] 
\right \}\]
of the plane spanned by ($\mathbf{e}$_1, $\mathbf{e}$_2).\\

We call this configuration the reference configuration of the body.
We may think of this configuration as the intersection with the ($\mathbf{e}$_1, $\mathbf{e}$_2)-plane of the reference configuration of the unstressed state of a slender 3-dimensional body in $\mathbb{E}$^3 that is symmetric about this plane. In this case, we limit our study here to deformations that preserve this symmetry. We call the section $\mathcal{R}$ a \emph{rod}.
\\\\
Let $\mathbf{r}$(s,t) : [0,1] × [t_1,t_2] → $\mathbb{R}$^2 be a smooth space curve describing the centerline of the rod.  The curve $\mathbf{r}$(s,t) should be chosen so that $\mathcal{R}$ represents a 'thickening' of $\mathbf{r}$(s,t). In particular, the lines normal to $\mathbf{r}$(s,t) should not intersect within $\mathcal{R}$. The intersection with $\mathcal{R}$ of the normal line through $\mathbf{r}$(s,t) is the \emph{material (cross) section} (at) s.  The vector 
\[ \mathbf{b}(s,t) := \mathbf{e}_3\times\mathbf{r}(s,t) \]
is a unit vector normal to $\mathbf{r}$(s,t); it lies along the section s in the ($\mathbf{e}$_1,$\mathbf{e}$_2)-plane.
\\\\
Notice that the coordinate s does not have to be arclength.
We work in a non-dimensionalized setting, that is, we assume that the length of the rod is equal to one. 

\subsection{Geometry of deformation}
Under the action of forces and couples,this body suffers deformations, which we assume to be planar. Thus the material point x$\mathbf{e}$_1 originally on the $\mathbf{e}$_1-axis goes to position $\mathbf{r}$(x,t) at time t and the cross-section  $\{$
$x\mathbf{e}_1 + y\mathbf{e}_2:y \in[-h,h]$ 
$\}$ goes into a curve containing $\mathbf{r}$(x,t). We assume that this curve is constrained so that its configuration can be characterized by the unit vector $\mathbf{b}$(s,t).
\\\\
An exact description of such a deformation would require partial differential equations with two independent spatial variables. We use the thinness of the body to motivate the construction of a simpler theory governed by ordinary differential equations in s. It is clear that if we know $\mathbf{r}$, we know the gross shape of the deformed body. In the model we employ, the planar version of the special Cosserat theory of rods, we find ordinary differential equations not only for $\mathbf{r}$ but also for the unit vector field b, which characterizes some average ori- entation of the deformed cross section. 
\\\\
The theory we now formulate, stands on its own as a coherent independent mathematical model of the 2-dimensional deformation of thin bodies.
\\\\
$\mathbf{Geometry}\,\,\mathbf{of}\,\,\mathbf{deformation}.$ A \emph{planar configuration} of a \emph{special Cosserat rod} is defined by a pair of vector-valued functions
\[[0,1] × [t_1,t_2] \in (s,t) → \mathbf{r}(s,t),\,\mathbf{b}(s,t)\,\,\in\,\,span(\mathbf{e}_11,\mathbf{e}_2)\]
where $\mathbf{b}(s,t)$ is a unit vector, called \emph{the director} at s. We can accordingly
represent $\mathbf{b}$ and the vector 
\[ \mathbf{a} := −\mathbf{e}_1 \times \mathbf{b}\]
by 
\[ \mathbf{a}=cos\theta\mathbf{e}_1+sin\theta\mathbf{e}_2,\,\,\mathbf{b}=−sin\theta\mathbf{e}_1+cos\theta\mathbf{e}_2.\]
Thus a configuration can be alternatively defined by $\mathbf{r}$ and $\theta$.
\\\\
Since the basis $\{\mathbf{a},\mathbf{b}\}$ is natural for the intrinsic description of deformation, we decompose all vector-valued functions with respect to it. In
particular, we set
\[ \mathbf{r}_s(s,t) =: \nu\mathbf{a} + \eta\mathbf{b}. \]
The functions
\[ \nu,\,\,\eta,\,\,\mu:= \theta_s \]
are the strain variables corresponding to the configuration. 



\end{document}
  
  
  
  
  
  
  
  
  
  
  
  
  
  
  
  
  
  
  
  
  
  
  
  
  
  
  
  
  
  
  
  
  
  
  
  
  
  
  
  
  
  
  
  
  
  
  
  