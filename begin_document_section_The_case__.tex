\begin{document}
\section{The case of the inextensible and unshearable Beck's Column immersed in a viscous fluid }

\\
If an \emph{elastic} rod is constrained so that $\eta$ = 0 in all circumstances, then the rod is said to be \emph{unshearable}.
We construct such a theory by simply defining $\hat{\eta}$=0, where ($\hat{\nu},\hat{\eta},\hat{\mu}$) is the global inverse of ($\hat{N},\hat{H},\hat{M}$). Actually, it can be easily shown that under the right assumptions (Per Luca: da aggiungerle e sviluppare l'argomento) the constitutive equations 
\[  N (s,t)=  \hat{N}(\nu (s,t),\eta (s,t))
\]
\[  H (s,t)=  \hat{H}(\nu (s,t),\eta (s,t)) 
\]
\[  M (s,t)=  \hat{M}(\nu (s,t),\eta (s,t), \mu(s,t))
\]
that define an elastic rod, are equivalent to the constitutive equations
\[  \nu (s,t)=  \hat{\nu}(N(s,t),H(s,t))
\]
\[  \eta (s,t)=  \hat{\eta}(N(s,t),H(s,t)) 
\]
\[  \mu (s,t)=  \hat{\nu}(N(s,t), H(s,t), M(s,t)).
\]
In this case, H, which is the Lagrange multiplier corresponding to this constraint, is not defined constitutively by any version of the above equations. It is just an unknown of any problem in which it appears.
\\
Likewise, if the rod is constrained so that $\nu$=1, then the rod is said to be \emph{inextensible}, and N is (a Lagrange multiplier) not defined constitutively.\\
The \emph{elastica} theory that stays behind our framework is based on these two constraints and on the constitutive equation that says that M is linear in the change in $\mu$:
\[
M(s,t) = C(\mu(s,t) − \alpha(s,t)),
\]
where C is the \emph{bending stiffness} and $\alpha$ is the \emph{spontaneous curvature}. We call M the \emph{active moment}. We introduce therefore the elastic potential density 
\[ \mathcal{U} = \frac{C}{2}(\mu(s,t) − \alpha(s,t))^2,\]
that models the fact that we are dealing with an inextensible and unshearable elastic medium which can actively vary its spontaneous curvature.
\\\\
In this case we don't have internal energy dissipation since we are no more assuming the column to be viscoelastic, but we do have external energy dissipation due to the nonconservative forces of the viscous fluid.
\\\\
The kinetic energy density and the external energy dissipation density expressed in terms of virtual works remain the same:
\[
\mathcal{T} = \frac{\rho S}{2}\mathbf{r}_t \cdot \mathbf{r}_t + \frac{\rho I}{2}\theta_t^2
\]
%%
\[ 
\delta W_{fluid}= -(\xi_\parallel\left|\mathbf{r}_t\cdot\mathbf{a}\right|^2 + \xi_\perp \left|\mathbf{r}_t\cdot\mathbf{b}\right|^2),
\]

\subsection{Geometry of deformation}
(Considerations on the new simplified configuration: => $\mathbf{r}_s$ = $\mathbf{a}$ ... Use of the exact Resistive force theory  ... etc etc...)
\\\\
\subsection{Derivation of the Equations of Motion}
We derive the equations of motion through Hamilton's Principle for a constrained system that involves both conservative and nonconservative forces. A solution ($\mathbf{r}$,$\theta$) must satisfy:

\[\int_{t_1}^{t_2} \int_{0}^{1} \delta (\mathcal{T}-\mathcal{U}) + \delta W_{fluid} + \delta \mathcal{C} \, ds \,dt
\]
for every perturbation $(\delta \mathbf{r}, \delta \theta)$ defined on $[0,1]x[t_1,t_2]$. 
\\\\
If we replace the value of each term, and define $\tilde{\mathbf{T}}$ = $\tilde{N}\mathbf{a}$ + $\tilde{H}\mathbf{b}$  the equations of motion finally read:

\[ \tilde{\mathbf{T}}_s - (\xi_\parallel (\mathbf{r}_t\cdot\mathbf{a})\mathbf{a} + \xi_\perp (\mathbf{r}_t\cdot\mathbf{b})\mathbf{b}) = \rho S \mathbf{r}_{tt}\\

M_s + \mathbf{e}_3 \cdot(\mathbf{r}_s \times \mathbf{\tilde{T}}) =\rho I \theta_{tt}.
\]
\\\\
We derive the componential form of the equations of motion by differentiating the first equation with respect to s.
Moreover, if we consider the equation for the active moment and the two equations on the constraints, we finally get six equations of motion in six variables, ($\theta,\nu,\eta$, M, $\tilde{N},\tilde{H}$).
\begin{align}
M_s + \nu \tilde{H} -\eta \tilde{N} - \rho I \theta_{tt}
&= 0\\
\tilde{N}_{ss} - 2\theta_s\tilde{H}_s - \tilde{H}\theta_{ss}- \tilde{N}\theta_s^2
- \xi_{\parallel}\nu_t + \xi_{\parallel}\eta\theta_t - \rho S \nu_{tt} + 2\rho S \theta_t\eta_t + \rho S \eta\theta_{tt} + \rho S \nu\theta_t^2
&= 0\\
\tilde{H}_{ss} + 2\theta_s\tilde{N}_s + \tilde{N}\theta_{ss} - \tilde{H}\theta_s^2  - \xi_{\perp}\nu\theta_t - \xi_{\perp}\eta_t
 - \rho S \eta_{tt} - 2\rho S \theta_t\nu_t -\rho S \nu\theta_{tt} + \rho S \eta\theta_t^2
&= 0\\
M - C(\theta_s -\alpha)  &= 0\\
\tilde{N}(\nu - 1) &= 0\\
\tilde{H}\eta &= 0
\end{align}

\end{document}