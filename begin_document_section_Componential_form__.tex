\begin{document}

\section{Componential form of the equations of motions}\label{componential-form-of-the-equations-of-motion}
It turns convenient to derive the componential form of the equations of motion by differentiating the first equation with respect to s.
Notice that the second equations has only one component along $\mathbf{e}_3$.
We begin by writing the components of $\mathbf{T}_s$, $\mathbf{F}_{extart}$ and $\rho$S$\mathbf{r}_{tt}$ along $\mathbf{a}$ and $\mathbf{b}$.
\begin{equation}
\begin{split}
\mathbf{T}=N\mathbf{a}+H\mathbf{b}
\end{spli}
\end{equation}
%%
\begin{equation}
\begin{split}
\mathbf{T}_{ss}& =((N\mathbf{a}+H\mathbf{b})_s)_s\\
               & =((N_s\mathbf{a}+ N\mathbf{a}_s + H_s\mathbf{b} + H\mathbf{b}_s)_s\\
               & = N_{ss}\mathbf{a} + N_s\mathbf{a}_s + N_s\mathbf{a}_s + N\mathbf{a}_{ss} + H_{ss}\mathbf{b} + H_s\mathbf{b}_s + H_s\mathbf{b}_s + H\mathbf{b}_{ss}
\end{split}
\end{equation}

 
 
 
 
 
\end{document} 
  
  
  
  
  
  
  
  