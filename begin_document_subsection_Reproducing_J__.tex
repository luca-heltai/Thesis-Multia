\begin{document}
\subsection{Reproducing Jülicher}
\\
\cite{julicher} studies the dynamics of an elastic rod-like filament in two dimensions, driven by internally generated forces. This situations is motivated by cilia and flagella which contain an axoneme. We mimic the formulation in \cite{julicher} with a discretization in weak form and with a spontaneous curvature, instead of the internal force. In the following sections we show how this study can be equivalently derived by considering the problem of the inextensible and unshearable Beck’s Column when we neglect the inertial terms.
\\\\
\subsubsection{Neglecting the inertial terms}
If we don't consider the inertial terms in the problem of the inextensible and unshearable Beck’s Column, then the equations of motion become:

\[ \tilde{\mathbf{T}}_s - (\xi_\parallel (\mathbf{r}_t\cdot\mathbf{a})\mathbf{a} + \xi_\perp (\mathbf{r}_t\cdot\mathbf{b})\mathbf{b}) = 0\\

M_s + \mathbf{e}_3 \cdot(\mathbf{r}_s \times \mathbf{\tilde{T}}) = 0.
\]
We differentiate the first equation with respect to s and derive the following componential form:
\begin{align}
M_s + \nu \tilde{H} -\eta \tilde{N} 
&= 0\\
\tilde{N}_{ss} - 2\theta_s\tilde{H}_s - \tilde{H}\theta_{ss}- \tilde{N}\theta_s^2
- \xi_{\parallel}\nu_t + \xi_{\parallel}\eta\theta_t &= 0\\
\tilde{H}_{ss} + 2\theta_s\tilde{N}_s + \tilde{N}\theta_{ss} - \tilde{H}\theta_s^2  - \xi_{\perp}\nu\theta_t - \xi_{\perp}\eta_t
&= 0\\
M - C(\theta_s -\alpha)  &= 0\\
\tilde{N}(\nu - 1) &= 0\\
\tilde{H}\eta &= 0
\end{align}

They yield the following system of the discrete equations of motion in matrix form:

\[
\begin{pmatrix}
0 & 0 & 0 & 0 & 0 & 0 \\
\xi_{\parallel}\mathbb{M}^\bar{\eta} & - \xi_{\parallel}\mathbb{M} & 0 & 0 & 0 & 0 \\
-\xi_{\perp} \mathbb{M}^\bar{\nu} & 0 & - \xi_{\perp}\mathbb{M} & 0 & 0 & 0 \\
0 & 0 & 0 & 0 & 0 & 0 \\
0 & 0 & 0 & 0 & 0 & 0 \\
0 & 0 & 0 & 0 & 0 & 0
\end{pmatrix}
\begin{pmatrix}
\theta_{t}\\
\nu_{t}\\
\eta_{t}\\
M_{t}\\
\tilde{N}_{t}\\
\tilde{H}_{t}
\end{pmatrix}\\
+ \begin{pmatrix}
0 & 0 & 0 & \mathbb{B} & -\mathbb{M}^{\bar{\eta}} & \mathbb{M}^{\bar{\nu}} \\
0 & 0 & 0 & 0 & -\mathbb{A} - \mathbb{M}^{\bar{\theta}_s^2} & -\mathbb{B}^{\bar{\theta_s}} + \mathbb{B}^{\bar{\theta_s}T}\\
0 & 0 & 0 & 0 &  \mathbb{B}^{\bar{\theta_s}} - \mathbb{B}^{\bar{\theta_s}T} & -\mathbb{A} - \mathbb{M}^{\bar{\theta}_s^2} \\
-C\mathbb{B} & 0 & 0 & \mathbb{M} & 0 & 0 \\
0 & 0 & 0 & 0 & \mathbb{M}^{(\bar{\nu}-1)} & 0 \\
0 & 0 & 0 & 0 & 0 & \mathbb{M}^{\bar{\eta}}
\end{pmatrix}
\begin{pmatrix}
\theta\\
\nu\\
\eta\\
M\\
\tilde{N}\\
\tilde{H}
\end{pmatrix}
\\
= \begin{pmatrix}
0\\
\langle \tilde{N}_s - \bar{\theta}_s \tilde{H}, \delta\nu \rangle \\
\langle \bar{\theta}_s \tilde{N} + \tilde{H}_s,\delta\eta \rangle \\
-C\mathbb{M}^{\alpha} - \langle C\theta_s,\delta M \rangle\\
0 \\
0
\end{pmatrix}
\]





\end{document}