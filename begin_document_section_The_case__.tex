\begin{document}
\section{The case of the inextensible and unshearable Beck's Column}


\subsection{Governing equations}

Consider an \emph{elastic} rod having consitutive equations of the form:
\[  N (s,t)=  \hat{N}(\nu (s,t),\eta (s,t)) 
\]
\[  H (s,t)=  \hat{H}(\nu (s,t),\eta (s,t)) 
\]
\[  M (s,t)=  \hat{M}(\nu (s,t),\eta (s,t), \mu(s,t)).
\]

We begin by imposing the constraints that the rod is inextensible and unsharable:
\begin{eqnarray*}
\nu(s,t)=1 \\ 
\eta(s,t)=0.
\end{eqnarray*}
Let $\bar{N}=\bar{N}$(s,t) and $\bar{H}=\bar{H}$(s,t) denote the \emph{Lagrange multipliers} representing the reactive internal tractions (axial tension and shear force, respectively) enforcing the constraints.
We give them the same letters as the components of contact force, because morally, from a physic point of view they have the same meaning.
We define
\[
\mathcal{C}(s,t) = \bar{N}(\nu-1) + \bar{H}\eta
\]
and consider
\[\int_{t_1}^{t_2} \int_{0}^{1} \delta \mathcal{C}\, ds \, dt = \int_{t_1}^{t_2} \int_{0}^{1} ((\bar{N}\mathbf{a})_s + (\bar{H}\mathbf{b})_s), \nu\bar{H}-\eta\bar{N})\cdot(\delta \mathbf{r},\delta\theta)
\,ds \, dt.
\]
It can be trivially shown by replacing the equations of the constraints that the elastic potential density becomes
\[ \mathcal{U}^{constr}(\mathbf{y}) = \frac{\cB}{2} (\mu -\alpha)^2,
\]
the internal dissipative energy density becomes
\[ \mathcal{D}_{diss}^{constr}(\mathbf{y}_t) = \frac{\tilde{C}}{2}\left|\mu_t\right|^2,
\]
while the kinetic energy density and the external energy dissipation density remain the same.
The Hamilton's Priciple then states
\[\int_{t_1}^{t_2} \int_{0}^{1} \delta (\mathcal{T}-\mathcal{U^{constr}}) + \delta \mathcal{C} ds \, dt
+ \int_{t_1}^{t_2} \int_{0}^{1} (\mathbf{Q}_{diss}^{constr} +
\mathbf{Q}_{ext})\cdot (\delta \mathbf{r},\delta\theta)
\, ds \, dt = 0
\]
for every perturbation $\delta \mathbf{y}$ defined on $[0,1]x[t_1,t_2]$ and vanishing at its boundary. By replacing the value of each term, we find
\[\int_{t_1}^{t_2} \int_{0}^{1} \delta (\mathcal{T}-\mathcal{U^{constr}}) + \delta \mathcal{C} ds \, dt
+ \int_{t_1}^{t_2} \int_{0}^{1} (\mathbf{Q}_{diss}^{constr} +
\mathbf{Q}_{ext})\cdot (\delta \mathbf{r},\delta\theta)
\, ds \, dt =
\int_{t_1}^{t_2} \int_{0}^{1}
         -(\ro S \mathbf{r}_{tt}
         +(\cA(\nu -\nu_0)_s,\cC(\eta -\eta_0)_s)\\
         + (\tilde_{A}(\nu_t)_s,\tilde_{B}(\eta_t)_s)
         - (\xi_\parallel (\mathbf{r}_t\cdot\mathbf{\tau})\mathbf{\tau} + \xi_\perp (\mathbf{r}_t\cdot\mathbf{P})\mathbf{P} + (\bar{N}\mathbf{a})_s + (\bar{H}\mathbf{b})_s) )
         \cdot\delta\mathbf{r}\,ds\,dt
         + \int_{t_1}^{t_2} \int_{0}^{1}
         (-\ro I \theta_{tt}
         + \cB(\mu-\alpha)_s 
         + \tilde_{C}(\mu_t)_s)
         +\nu\bar{N}-\eta\bar{H})
         \cdot\delta\theta \,ds\,dt
\]
and the equations of motion become
\[ (\bar{N}\mathbf{a} + \bar{H}\mathbf{b})_s - (\xi_\parallel (\mathbf{r}_t\cdot\mathbf{\tau})\mathbf{\tau} + \xi_\perp (\mathbf{r}_t\cdot\mathbf{P})\mathbf{P}) = \rho S \mathbf{r}_{tt}\\
M_s + \mathbf{e}_3\cdot(\mathbf{r}_s\cdot(\bar{N}\mathbf{a} + \bar{H}\mathbf{b}))=\rho I \theta_{tt}
\]
\subsection{Stability Analysis and Hopf Bifurcation}
\end{document}