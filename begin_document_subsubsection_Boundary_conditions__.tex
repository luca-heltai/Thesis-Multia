\begin{document}
\subsubsection{Boundary conditions}
In this section we prescribe the boundary conditions that the deformed configuration of our problem must satisfy.\\\\
We assume that the column has one end welded to a rigid support. We prescribe the position of the end s_1 = 0 by requiring that there be the given vector $\mathbf{r}_1$ such that
\[ \mathbf{r}(0,t) = \mathbf{r}_1.\]
We assume that $\mathbf{r}_1$ = $\mathbf{0}.$
In this case, we do not prescribe the force $\mathbf{T}(0,t)$ acting at the end, so that it remains free to accommodate (45). Moreover, we have assumed that the other end is subjected to a \emph{compressive follower force}. Such a force has the defining property that its magnitude is fixed but its direction must remain normal to the cross-section of the column at the end where this force is applied. Since we have assumed that the direction of the cross-section is determined by the vector $\mathbf{b}$, then we prescribe the force acting at the end s_2 = 1 by requiring that 
\[ \mathbf{T}(1,t) = -\lambda \mathbf{a}(1,t), \]
where $\lambda$ is a constant.In this case, we do not prescribe $\mathbf{r}(1).$ We observe that for equilibrium problems, one is not free to prescribe forces at both ends in an arbitrary way because it is necessary, but not sufficient, for equilibrium that the resultant force on the rod vanish.



\end{document}