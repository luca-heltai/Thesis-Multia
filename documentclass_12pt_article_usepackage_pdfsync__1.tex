\documentclass[12pt]{article}

\usepackage{pdfsync}
\usepackage{amsmath}
\usepackage{graphicx}

\graphicspath{{./figures/}}

%%%%%%%%%%%%%%%%%%%%%%%%%%%%%%%%%%%%%%%%%%%%%%%%%%%%%%%%%%%%
\renewcommand{\d}{\mathrm{d}}
\newcommand{\vv}[1]{\boldsymbol{#1}}
\newcommand{\cB}{B}
\newcommand{\cA}{A}
\newcommand{\cC}{C}


%%%%%%%%%%%%%%%%%%%%%%%%%%%%%%%%%%%%%%%%%%%%%%%%%%%%%%%%%%%%


\begin{document}
\section{Derivation of the Equations of Motion
}\label{Derivation of the Equations of Motion
}
\subsection{Hamilton's Principle}
\subsection{Derivation of the Equations of Motion}
Under the action of forces, the elastic rod-like filament suffers deformations, which we assume to be planar. Whenever these deformations occur, it will be a contact force \textbf{n}(s,t) exerted by the material of [s,1] on that of [0,s). \textbf{n}(s,t) is assumed to lie to the (\textbf{e_1,e_2})-plane, so that it has the form
\[  \textbf{n}(s,t)= \textbf{N} (s,t) \textbf{a}(s,t) + \textbf{H} (s,t) \textbf{b} (s,t).
\]
Let \textbf{m}(s,t) be the contact couple exerted by the material of [s,1] on tha of [0,s). We assume that \textbf{m} is perpendicular to the (\textbf{e_1,e_2})-plane so that it has the form
\[ 
\textbf{m}(s,t) = M(s,t)\textbf{e}_3.
\]

Furthermore, we assume that the material of the rod is viscoelastic,and model this with constitutive equations of the form

We model this by introducing the elastic potential density as
\[ \mathcal{U}(\textbf{y}) = \frac{\cA}{2} (\nu - \nu_0)^2
+\frac{\cC}{2} (\eta - \eta_0 )^2
+\frac{\cB}{2} (\mu -\alpha)^2 
\]
where
\end{document}
  
  
  
  
  
  
  
  