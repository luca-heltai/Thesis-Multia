\begin{document}
\subsection{Hopf Bifurcation in the classical Beck's Problem}
\subsubsection{Estimates for linear equations}
A partial differential equation is said to be \emph{linear} if it is linear with respect to the unknown function and its derivatives that appear in it.
\subsubsection{Local existence and estimates for quasilinear equations}
We use now the estimates from last section to prove the existence of solutions for our quasilinear system
\[
\begin{align}
M_s + \nu H -\eta N 
&= \rho I \theta_{tt}\\
N_{ss} - 2\theta_sH_s - H\theta_{ss}- N\theta_s^2 
- \xi_{\parallel}\nu_t + \xi_{\parallel}\eta\theta_t
&= \rho S (\nu_{tt} - 2 \theta_t\eta_t - \eta\theta_{tt} - \nu\theta_t^2)\\
H_{ss} + 2\theta_sN_s + N\theta_{ss} - H\theta_s^2 
- \xi_{\perp}\nu\theta_t - \xi_{\perp}\eta_t
&= \rho S (\eta_{tt} + 2\theta_t\nu_t + \nu\theta_{tt} - \eta\theta_t^2).
\end{align}
\]
(We recall that a partial differential equation is said to be \emph{quasilinear} if it is linear with respect to all the highest order derivatives of the unknown function. Here: we have three second order quasilinear partial differential equations, they are in fact linear with respect to $\theta$_{tt}, $\nu$_{tt} and $\eta$_{tt}.)
with boundary conditions
\[
\mathbf{r}(0,t)= 0,\\
\theta(0,t) = 0,\\
\mathbf{T}(1,t) = -\lambda \mathbf{a}(1,t),\\
M(1,t) = 0\\
\]
and initial conditions close to the equilibrium solutions.
\\\\
We want now to use the estimates we derived for linear equations. We convert our equations of motion into a system with only first order t-derivatives by introducing
\[
\cdot\theta = \theta_t\dot\nu = \nu_t, \dot\eta = \eta_t
\]
and rewrite the equations 
\[
\]




\subsection{Hopf Bifurcation in the generalized Beck's Problem}
\end{document}