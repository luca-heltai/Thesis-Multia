\documentclass[12pt]{article}

\usepackage{pdfsync}
\usepackage{amsmath}
\usepackage{graphicx}
\usepackage{textcomp}
\usepackage[T1]{fontenc}

\graphicspath{{./figures/}}

%%%%%%%%%%%%%%%%%%%%%%%%%%%%%%%%%%%%%%%%%%%%%%%%%%%%%%%%%%%%
\renewcommand{\d}{\mathrm{d}}
\newcommand{\vv}[1]{\boldsymbol{#1}}
\newcommand{\cB}{B}
\newcommand{\cA}{A}
\newcommand{\cC}{C}



%%%%%%%%%%%%%%%%%%%%%%%%%%%%%%%%%%%%%%%%%%%%%%%%%%%%%%%%%%%%


\begin{document}
We briefly summarize the basic aspects of the special Cosserat theory of rods using a notation adopted from Antman. Generally, a rod is a fiber–like elastic body, i.e. it is possible to specify a family of cross–sections which have small proportions compared to the length of the rod. This suggests a mathematical model of a rod in terms of a spacecurve corresponding to its centerline and a director frame which defines the orientation of the local
cross–section plane.
Throughout this work, we denote three–dimensional Euclidean space by ($\mathbf{E}$, ⟨ · , · ⟩) and choose a fixed right–handed orthonormal triple ($\mathbf{e}$_1, \mathbf{e}_2, \mathbf{e}_3) of basis vectors.



\end{document}
  
  
  