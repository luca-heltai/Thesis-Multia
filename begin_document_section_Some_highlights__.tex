\begin{document}
\section{Some highlights of Hamilton's Principle}
\\
Consider a system of particles in $\mathbb{E}^3$ whose position, or \emph{configuration}, can be described by a set of independent generalized coordinates $\mathbf{q}$_{k}, k = 1, 2, . . . ,K. Let t_1 and t_2 be fixed times, with t_1 < t_2, and suppose that $\mathbf{q}$(t_1) and $\mathbf{q}$(t_2) are prescribed.\\
An \emph{admissible} motion of the system will be defined to be a set of functions $\mathbf{q}$_{k}(t), k = 1, 2, . . .,K, which satisfy the prescribed values at t_1 and t_2 and are C^2 on [t_1, t_2].\\
We translate in mathematical terms the fact that there might be different configurations by introducing a \emph{perturbation} $\delta\mathbf{q}$_k, which is an arbitrary C^2 function on [t_1,t_2] subject to the requirements that $\delta\mathbf{q}$_k(t_1)= 0 and $\delta\mathbf{q}$_k(t_2)=0.
\\
Let the admissible \emph{comparison motion} of the system be defined by 
\[ \mathbf{q}_k^{\ast}(t,\epsilon) = \mathbf{q}_k(t) + \epsilon\eta\mathbf{q}_k(t) k = 1, 2, . . .,K \]
It will be convenient to introduce the notation
\[  \delta(\cdot) = \left[ \frac{\partial}{\partial\epsilon}(\cdot)^{\ast} \right]_{\epsilon=0} \]
The symbol $\delta(\cdot)$ is called the variation of the expression $(\cdot)$.
\\
Let it be assumed that the kinetic energy of the system, $\mathcal{T}$, can be expressed as a function of the generalized coordinates and their time derivatives, $\mathcal{T} = \mathcal{T}(\mathbf{q}_k,\cdot{\mathbf{q}}_k)$. This expression indicates that $\mathcal{T}$ may be a function of $\mathbf{q}$_k and $\mathbf{q}$_k for each value of k from 1 to K.
\\
If the system is subject to conservative forces, then its potential energy $\mathcal{U}$ can be expressed as a function of the generalized coordinates, $\mathcal{U}$=$\mathcal{U}$($\mathbf{q}$_k). 
\\
Each of the second partial derivatives of $\mathcal{T}$ and each of the first partial derivatives of $\mathcal{U}$ will be assumed to exist and to be continuous.
\\
We define the \emph{Lagrangian} of the system
\[ \mathcal{L}=\mathcal{T}-\mathcal{U}, \mathcal{L}=\mathcal{L}(\mathbf{q},\dot{\mathbf{q}}_k).\]

\end{document}
  
  
  
  
  
  
  
  
  
  
  
  
  
  
  
  
  
  
  