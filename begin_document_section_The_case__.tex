\begin{document}
\section{The case of the inextensible and unshearable Beck's Column}

\\
If an \emph{elastic} rod is constrained so that $\eta$ = 0 in all circumstances, then the rod is said to be \emph{unshearable}.
We construct such a theory by simply defining $\hat{\eta}$=0, where ($\hat{\nu},\hat{\eta},\hat{\mu}$) is the global inverse of ($\hat{N},\hat{H},\hat{M}$). Actually, it can be easily shown that under the right assumptions (Per Luca: da aggiungerle e sviluppare l'argomento) the constitutive equations 
\[  N (s,t)=  \hat{N}(\nu (s,t),\eta (s,t))
\]
\[  H (s,t)=  \hat{H}(\nu (s,t),\eta (s,t)) 
\]
\[  M (s,t)=  \hat{M}(\nu (s,t),\eta (s,t), \mu(s,t))
\]
that define an elastic rod, are equivalent to the constitutive equations
\[  \nu (s,t)=  \hat{\nu}(N(s,t),H(s,t))
\]
\[  \eta (s,t)=  \hat{\eta}(N(s,t),H(s,t)) 
\]
\[  \mu (s,t)=  \hat{\nu}(N(s,t), H(s,t), M(s,t)).
\]
In this case, H, which is the Lagrange multiplier corresponding to this constraint, is not defined constitutively by any version of the above equations. It is just an unknown of any problem in which it appears.
\\
Likewise, if the rod is constrained so that $\nu$=1, then the rod is said to be \emph{inextensible}, and N is (a Lagrange multiplier) not defined constitutively.\\
The \emph{elastica} theory that stays behind our framework is based on these two constraints and on the constitutive equation that says that M is linear in the change in $\mu$:
\[
M(s,t) = C(\mu(s,t) − \alpha(s,t)),
\]
where C is the \emph{bending stiffness} and $\alpha$ is the \emph{spontaneous curvature}. We call M the \emph{active moment}. We introduce therefore the elastic potential density 
\[ \mathcal{U} = \frac{C}{2}(\mu(s,t) − \alpha(s,t))^2,\]
that models the fact that we are dealing with an inextensible and unshearable elastic medium which can actively vary its spontaneous curvature.
The kinetic energy density and the external energy dissipation density remain the same as in the viscoelastic case:
\[
\mathcal{T} = \frac{\rho S}{2}\mathbf{r}_t \cdot \mathbf{r}_t + \frac{\rho I}{2}\theta_t^2
\]
%%
\[ 
\delta W_{visc}= -(\xi_\parallel\left|\mathbf{r}_t\cdot\mathbf{a}\right|^2 + \xi_\perp \left|\mathbf{r}_t\cdot\mathbf{b}\right|^2),
\]


\subsection{Derivation of the Equations of Motion}
We derive the equations of motion through Hamilton's Principle for a constrained system that involves both conservative and nonconservative forces states:
The Hamilton's Priciple then states

\[\int_{t_1}^{t_2} \int_{0}^{1} \delta (\mathcal{T}-\mathcal{U}) + \delta W_{visc} + \delta \mathcal{C} \, ds \,dt
\]
for every perturbation $\delta \mathbf{y}$ defined on $[0,1]x[t_1,t_2]$. 
\\
If we replace the value of each term, and define $\mathbf{T}$ = $\tilde{N}\mathbf{a}$ + $\tilde{H}\mathbf{b}$  the equations of motion finally read:


\[ (\tilde{T}_s- (\xi_\parallel (\mathbf{r}_t\cdot\mathbf{a})\mathbf{a} + \xi_\perp (\mathbf{r}_t\cdot\mathbf{b})\mathbf{b}) = \rho S \mathbf{r}_{tt}\\

M_s + \mathbf{e}_3 \cdot(\mathbf{r}_s\cdot(\mathbf{\tilde{T}} =\rho I \theta_{tt}
\]
\subsection{Stability Analysis and Hopf Bifurcation}
\end{document}