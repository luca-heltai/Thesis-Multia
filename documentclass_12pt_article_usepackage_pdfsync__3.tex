\documentclass[12pt]{article}

\usepackage{pdfsync}
\usepackage{amsmath}
\usepackage{graphicx}
\usepackage{textcomp}
\usepackage[T1]{fontenc}

\graphicspath{{./figures/}}

%%%%%%%%%%%%%%%%%%%%%%%%%%%%%%%%%%%%%%%%%%%%%%%%%%%%%%%%%%%%
\renewcommand{\d}{\mathrm{d}}
\newcommand{\vv}[1]{\boldsymbol{#1}}
\newcommand{\cB}{B}
\newcommand{\cA}{A}
\newcommand{\cC}{C}



%%%%%%%%%%%%%%%%%%%%%%%%%%%%%%%%%%%%%%%%%%%%%%%%%%%%%%%%%%%%


\begin{document}
\section{Introduction}
In this work we explore the planar, nonlinear dynamics of an extensional shearable rod by using the simple Cosserat model. The method of Cosserat dynamics for elastic structures is employed since it can accommodate to a good approximation the nonlinear behavior of complex elastic structures composed of materials with different constitutive properties, variable geometry and damping characteristics. With arbitrary boundary conditions, the Cosserat theory is used to formulate a set of governing partial differential equations of motion in terms of the displacements and angular variables, describing the planar, nonlinear dynamics of an extensional rod. Bending about two principal axes, extension, shear and torsion are considered, and care is taken into account for all the nonlinear terms in the resulting equations.We derive, through variational principles, the equations of motion for two special cases:...

\section{Background on the special Cosserat model}
We briefly summarize the basic aspects of the special Cosserat theory of rods using a notation adopted from Antman. Generally, a rod is a fiber–like elastic body, i.e. it is possible to specify a family of cross–sections which have small proportions compared to the length of the rod. This suggests a mathematical model of a rod in terms of a spacecurve corresponding to its centerline and a director frame which defines the orientation of the local cross–section plane.
Throughout this work, we denote three–dimensional Euclidean space by ($\mathbb{E}$^3, ⟨ · , · ⟩) and choose a fixed right–handed orthonormal triple ($\mathbf{e}$_1, $\mathbf{e}$_2, $\mathbf{e}$_3) of basis vectors.

\subsection{Configuration}
Let a thin 2-dimensional body occupy a rectangular region
\[  \mathcal{R}= \left \{
x\mathbf{e}_1 + y\mathbf{e}_2: x \mathcal{2}[0,1] ,x \mathcal{2}[-h,h] 
\right \}\]
of the plane spanned by ($\mathbf{e}$_1, $\mathbf{e}$_2).\\

We call this configuration the reference configuration of the body.
We may think of this configuration as the intersection with the ($\mathbf{e}$_1, $\mathbf{e}$_2)-plane of the reference configuration of the unstressed state of a slender 3-dimensional body in $\mathbb{E}$^3 that is symmetric about this plane. In this case, we limit our study here to deformations that preserve this symmetry. We call the section $\mathcal{R}$ a \emph{rod}.
\\\\
Let $\mathbf{r}$(s,t) : [s_1,s_2] × [t_1,t_2] → $\mathbb{R}$^2 be a smooth space curve describing the centerline of the rod.  The curve $\mathbf{r}$(s,t) should be chosen so that $\mathcal{R}$ represents a 'thickening' of $\mathbf{r}$(s,t). In particular, the lines normal to $\mathbf{r}$(s,t) should not intersect within $\mathcal{R}$. The intersection with $\mathcal{R}$ of the normal line through $\mathbf{r}$(s,t) is the \emph{material (cross) section} (at) s.  The vector 
\[ \mathbf{b}(s,t) := \mathbf{e}_3\times\mathbf{r}(s,t) \]
is a unit vector normal to $\mathbf{r}$(s,t); it lies along the section s.
\\\\
Notice that the coordinate s does not have to be arclength.
We work in a non-dimensionalized setting, that is, we assume that the length of the rod is equal to one. 

\section{Geometry of deformation}
Under the action of forces and couples,this body suffers deformations, which we assume to be planar. Thus the material point x\mathbf${e}$_1 originally on the \mathbf${e}$_1-axis goes to position $\mathbf{r}$(x,t) at time t and the cross-section  $\left \{$
$x\mathbf{e}_1 + y\mathbf{e}_2:y \mathcal{2}[-h,h]$ 
$\right \}$ goes into a curve containing $\mathbf{r}$(x,t).



\end{document}
  
  
  
  
  
  
  
  
  
  
  
  
  
  
  
  
  
  