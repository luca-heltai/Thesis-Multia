\begin{document}
\subsubsection{Boundary conditions in classical Beck's problem}
In this section we prescribe the boundary conditions that the deformed configuration of our problem must satisfy. We take s_0=0 and s_1=1.\\\\
We assume that the column has one end welded to a rigid support. We prescribe the position of the end s_0 = 0 by requiring that there be the given vector $\mathbf{r}_0$ such that
\[ \mathbf{r}(0,t) = \mathbf{r}_0.\]
In this case, we do not prescribe the force $\mathbf{T}(0,t)$ acting at the end, so that it remains free to accommodate (45). Moreover, we have assumed that the other end is subjected to a \emph{compressive follower force}. Such a force has the defining property that its magnitude is fixed but its direction must remain normal to the cross-section of the column at the end where this force is applied. Since we have assumed that the direction of the cross-section is determined by the vector $\mathbf{b}$, then we prescribe the force acting at the end s_1 = 1 by requiring that 
\[ \mathbf{T}(1,t) = -\lambda \mathbf{a}(1,t), \]
where $\lambda$ is a constant. In this case, we do not prescribe $\mathbf{r}(1,t).$ We observe that for equilibrium problems, one is not free to prescribe forces at both ends in an arbitrary way because it is necessary, but not sufficient, for equilibrium that the resultant force on the rod vanish.
\\\\
We can prescribe the orientation of the end section $\mathbf{b}(0,t)$ by requiring that there be a given number $\theta_0$ such that
\[
\theta(0,t)=\theta_0
\]
This condition can be effected by welding the end section to a rigid wall whose normal makes an angle $\theta_0$ with $\mathbf{e}_1$.In this case, we do not prescribe M(0,t).
We can the bending couple at s_1=1 by requiring that there be a given number M_1 such that
\[ M(1,t) = M_1.\]
If we assume that $\mathbf{r}_0$ = $\mathbf{0},$ $\theta_0=0$ and M_1=0, we find the following boundary conditions for Beck's classical problem:
\[ 
\mathbf{r}(0,t) = \mathbf{0},\\
\theta(0,t) = 0\\
\mathbf{T}(1,t) = -\lambda \mathbf{a}(1,t)\\
M(1,t) = 0.
\]
The condition $\mathbf{r}(0,t)$ = $\mathbf{0}$, implies the two scalar conditions
\[
\nu(0,t)= 0\\
\eta(0,t)= 0
\]
which are homogeneous Dirichlet conditions for the spaces of $\nu$ and $\eta$. $\theta$(0,t)=0 is a homogeneous Dirichlet condition for the space of $\theta$, too.
\end{document}