\begin{document}

\section{Discretization}\label{matrix-form-of-the-weak-formulation}

discretizzazione => QUINDI introduzione degli spazi Vh ....

Given an arbitrary variable $\gamma$, we’ll use the following shorthand notation:

\begin{equation}
  \label{eq:2}
  \begin{aligned}
   \mathbb{A}^\gamma_{ij} & = (\gamma N_j', N_i') &  = \int_0^1 \gamma(s) N_j'(s)
    N_i'(s) \, ds \\
     \mathbb{B}^\gamma_{ij} & = (\gamma N_j', N_i) & = \int_0^1 \gamma(s) N_j'(s)
    N_i(s) \, ds \\
     \mathbb{M}^\gamma_{ij} & = (\gamma N_j, N_i) & = \int_0^1 \gamma(s) N_j(s)
    N_i(s) \, ds 
  \end{aligned}
\end{equation}

(Anche questa riga che viene da scrivere meglio:)
If we restrict the functions $\theta, \nu, \eta$, H, N ,M to live in V_h, we
can rewrite the system of the discrete equations of motion in matrix form:

\[\begin{pmatrix}
-\rho I \mathbb{M} & 0 & 0 & 0 & 0 & 0 \\
\rho S \mathbb{M}^\bar{\eta} & -\rho S \mathbb{M} & 0 & 0 & 0 & 0 \\
-\rho S \mathbb{M}^\bar{\nu} & 0 & \rho S \mathbb{M} & 0 & 0 & 0 \\
0 & 0 & 0 & 0 & 0 & 0 \\
0 & 0 & 0 & 0 & 0 & 0 \\
0 & 0 & 0 & 0 & 0 & 0
\end{pmatrix}
\begin{pmatrix}
\theta_{tt}\\
\nu_{tt}\\
\eta_{tt}\\
M_{tt}\\
N_{tt}\\
H_{tt}
\end{pmatrix}\\

+ \begin{pmatrix}
0 & 0 & 0 & 0 & 0 & 0 \\
\rho S \mathbb{M}^{\bar{\nu}\bar{\theta}_t} + \xi_{\parallel}\mathbb{M}^\bar{\eta} & - \xi_{\parallel}\mathbb{M} & 2\rho S \mathbb{M}^{\bar{\theta}_t} & 0 & 0 & 0 \\
\rho S \mathbb{M}^{\bar{\eta}\bar{\theta}_t} -\xi_{\perp} \mathbb{M}^\bar{\nu} & -2\rho S \mathbb{M}^\bar{\theta}_t & - \xi_{perp}\mathbb{M} & 0 & 0 & 0 \\
-\tilde{C}\mathbb{B} & 0 & 0 & 0 & 0 & 0 \\
0 & -\tilde{A}\mathbb{M} & 0 & 0 & 0 & 0 \\
0 & 0 & -\tilde{B}\mathbb{M} & 0 & 0 & 0
\end{pmatrix}
\begin{pmatrix}
\theta_{t}\\
\nu_{t}\\
\eta_{t}\\
M_{t}\\
N_{t}\\
H_{t}
\end{pmatrix}\\
\]

\end{document}
  
  
  
  
  
  
  
  
  
  
  
  
  
  
  