\documentclass{article}
\usepackage{amsmath}
\begin{document}

\subsubsection{Componential form of the equations of motion}\label{componential-form-of-the-equations-of-motion}

It is convenient to derive the componential form of the equations of motion by differentiating the first equation with respect to s.
Notice that the second equation has only one component along $\mathbf{e}_3$.
We begin by writing the components of $\mathbf{T}_s$, $\mathbf{F}_{extart}$ and $\rho$S$\mathbf{r}_{tt}$ along $\mathbf{a}$ and $\mathbf{b}$. From $\mathbf{T}=N\mathbf{a}+H\mathbf{b}$ we get
\begin{equation}
\begin{split}
\mathbf{T}_{ss}& =((N\mathbf{a}+H\mathbf{b})_s)_s\\
               & =((N_s\mathbf{a}+ N\mathbf{a}_s + H_s\mathbf{b} + H\mathbf{b}_s)_s\\
               & = N_{ss}\mathbf{a} + N_s\mathbf{a}_s + N_s\mathbf{a}_s + N\mathbf{a}_{ss} + H_{ss}\mathbf{b} + H_s\mathbf{b}_s + H_s\mathbf{b}_s + H\mathbf{b}_{ss}.
\end{split}
\end{equation}
Exploiting the fact that $\mathbf{a}=(cos\theta, sen\theta)$ and $\mathbf{b}=(-sen\theta, cos\theta)$ we can express $\mathbf{a}_s$, $\mathbf{a}_{ss}, \mathbf{b}_s, \mathbf{b}_{ss}$ in terms of $\mathbf{a}$ and $\mathbf{b}$ only :

\begin{align}
\mathbf{a}   & = (cos\theta,sen\theta)                        &\mathbf{b}& = (-sen\theta, cos\theta)\\\\
\mathbf{a}_s & = ((-sen\theta)\theta_s, (cos\theta)\theta_s)  &\mathbf{b}_s& = ((-cos\theta)\theta_s, (-sen\theta)\theta_s)\\
             & = \theta_s(-sen\theta, cos\theta)              & &= \theta_s(-(cos\theta, sen\theta))\\
             & =\theta_s\mathbf{b}                            & &= -\theta_s\mathbf{a}\\\\
\mathbf{a}_{ss}& = \theta_{ss}\mathbf{b}+\theta_s\mathbf{b}_s & 
\mathbf{b}_{ss}& = -\theta_{ss}\mathbf{a}+\theta_s\mathbf{a}_s\\
               & = \theta_{ss}\mathbf{b} - \theta_s\theta_s\mathbf{a} & &= -\theta_{ss}\mathbf{a} -\theta_s\theta_s\mathbf{b}\\
               & = \theta_{ss}\mathbf{b} - \theta_s^2\mathbf{a} & &= -\theta_{ss}\mathbf{a} -\theta_s^2\mathbf{b}.
\end{align}

By substituting, we find that
\begin{equation}
\begin{split}
\mathbf{T}_{ss}& = N_{ss}\mathbf{a} + 2N_s(\theta_s\mathbf{b}) + N(\theta_{ss}\mathbf{b}-\theta_s^2\mathbf{a}) + H_{ss}\mathbf{b} + 2H_s(-\theta_s\mathbf{a}) + H(-\theta_{ss}\mathbf{a}-\theta_s^2\mathbf{b})\\
               & = N_{ss}\mathbf{a} + 2\theta_sN_s\mathbf{b}- N\theta_s^2\mathbf{a} + N\theta_ss\mathbf{b} + H_{ss}\mathbf{b} -2\theta_sH_s\mathbf{a}-H\theta_s^2\mathbf{b}-H\theta_{ss}\mathbf{a}\\
               & = (N_{ss} -2\theta_sH_s - H\theta_{ss}-N\theta_s^2)\mathbf{a} + (H_{ss}+2\theta_sN_s + N\theta_{ss} -H\theta_s^2)\mathbf{b}.
\end{split}
\end{equation}
Doing the same for $(\rho S\mathbf{r}_{tt})_s$ and exploiting the fact that 
\begin{align}
\mathbf{a}_t    &=\theta_t\mathbf{b} & \mathbf{b}_t &= -\theta_t\mathbf{a}\\
\mathbf{a}_{tt} &=\theta_{tt}\mathbf{b}-\theta_t^2\mathbf{a} & \mathbf{b}_{tt} &= -\theta_{tt}\mathbf{a}-\theta_t^2\mathbf{b} 
\end{align}
yields
\begin{equation}
\begin{split}
\rho S(\mathbf{r}_{tt})_s &= \rho S (\mathbf{r}_s)_{tt}\\
                          &= \rho S ((\nu\mathbf{a} + \eta\mathbf{b})_t)_t\\
                          &= \rho S (\nu_t\mathbf{a} + \nu\mathbf{a}_t + \eta_t\mathbf{b} + \eta\mathbf{b}_t)_t\\
                          &= \rho S (\nu_{tt}\mathbf{a} + \nu_t\mathbf{a}_t + \nu_t\mathbf{a}_t + \nu\mathbf{a}_{tt} + \eta_{tt}\mathbf{b} + \eta_t\mathbf{b}_t + \eta_t\mathbf{b}_t + \eta\mathbf{b}_{tt})\\
                          &= \rho S (\nu_{tt}\mathbf{a} + 2\nu_t\mathbf{a}_t + \nu\mathbf{a}_{tt} + \eta_{tt}\mathbf{b} + 2\eta_t\mathbf{b}_t + \eta\mathbf{b}_{tt})\\
                          &= \rho S (\nu_{tt}\mathbf{a} + 2\nu_t(\theta_t\mathbf{b}) + \nu(\theta_{tt}\mathbf{b}-\nu\theta_t^2\mathbf{a}) + \eta_{tt}\mathbf{b} + 2\eta_t(-\theta_t\mathbf{a}) + \eta(-\theta_{tt}\mathbf{a}-\theta_t^2\mathbf{b}))\\
                          &= \rho S (\nu_{tt}\mathbf{a} + 2\nu_t\theta_t\mathbf{b} + \nu\theta_{tt}\mathbf{b}-\nu\theta_t^2\mathbf{a} + \eta_{tt}\mathbf{b} - 2\theta_t\eta_t\mathbf{a} - \eta\theta_{tt}\mathbf{a} - \eta\theta_t^2\mathbf{b})\\
                          &= \rho S \left[ (\nu_{tt}-\nu\theta_t^2 - 2\theta_t\eta_t -\eta\theta_{tt})\mathbf{a} + (\nu\theta_{tt} + 2\theta_t\nu_t + \eta_{tt} - \eta\theta_t^2)\mathbf{b} \right]\\
                          &= \rho S (\nu_{tt} - 2\theta_t\eta_t -\eta\theta_{tt} - nu\theta_t^2  )\mathbf{a}+ (\eta_{tt}+ 2\theta_t\nu_t + \nu\theta_{tt} - \eta\theta_t^2)\mathbf{b}).
\end{split}
\end{equation}
For the force applied by the fluid on the rod we get
\begin{equation}
\begin{split}
\mathbf{F}_{extart}_s 
                      &= -\left[\xi_{\parallel}(\mathbf{r}_t\cdot\mathbf{a})\mathbf{a}
+ \xi_{\perp}(\mathbf{r}_t\cdot\mathbf{b})\mathbf{b}\right]_s\\
                      &= -\xi_{\parallel}\left[(\mathbf{r}_t\cdot\mathbf{a})_s\mathbf{a} + (\mathbf{r}_t\cdot\mathbf{a})\mathbf{a}_s\right] - \xi_{\perp}\left[(\mathbf{r}_t\cdot\mathbf{b})_s\mathbf{b} + (\mathbf{r}_t\cdot\mathbf{b})\mathbf{b}_s\right]\\
                      &= -\xi_{\parallel}\left[(\mathbf{r}_t\cdot\mathbf{a})_s\mathbf{a} + (\mathbf{r}_t\cdot\mathbf{a})\theta_s\mathbf{b}\right] - \xi_{\perp}\left[(\mathbf{r}_t\cdot\mathbf{b})_s\mathbf{b} - (\mathbf{r}_t\cdot\mathbf{b})\theta_s\mathbf{a)\right]\\
                      &= -\xi_{\parallel}(\mathbf{r}_t\cdot\mathbf{a})_s\mathbf{a} - \xi_{\perp}(\mathbf{r}_t\cdot\mathbf{b})_s\mathbf{b}\\
                      &= -\xi_{\parallel}\left[(\mathbf{r}_t)_s\cdot\mathbf{a} + \mathbf{r}_t\cdot\mathbf{a}_s\right]\mathbf{a} - \xi_{\perp}\left[(\mathbf{r}_t)_s\cdot\mathbf{b} + \mathbf{r}_t\cdot\mathbf{b}_s\right]\mathbf{b}\\
                      &= -\xi_{\parallel}\left[(\mathbf{r}_s)_t\cdot\mathbf{a}\right]\mathbf{a} - \xi_{\perp}\left[(\mathbf{r}_s)_t\cdot\mathbf{b}\right]\mathbf{b}.
\end{split}
\end{equation}
%%
Along $\mathbf{a}$:
%%
\begin{equation}
\begin{split}
-\xi_{\parallel}\left[(\mathbf{r}_s)_t\cdot\mathbf{a}\right] &= -\xi_{\parallel}\left[(\nu\mathbf{a} + \eta\mathbf{b})_t\cdot\mathbf{a}\right]\\
&= -\xi_{\parallel}\left[(\nu_t\mathbf{a} +\nu\mathbf{a}_t + \eta_t\mathbf{b} + \eta\mathbf{b}_t)\cdot\mathbf{a}\right]\\
&= -\xi_{\parallel}\left[(\nu_t\mathbf{a} +\nu\theta_t\mathbf{b} + \eta_t\mathbf{b} - \eta\theta_t\mathbf{a})\cdot\mathbf{a}\right]\\
&= -\xi_{\parallel}(\nu_t - \eta\theta_t)\\
&= -\xi_{\parallel}\nu_t + \xi_{\parallel}\eta\theta_t.
\end{split}
\end{equation}
%%
Along $\mathbf{b}$:
\begin{equation}
\begin{split}
-\xi_{\perp}\left[(\mathbf{r}_s)_t\cdot\mathbf{a}\right] &= -\xi_{\perp}\left[(\nu\mathbf{a} + \eta\mathbf{b})_t\cdot\mathbf{a}\right]\\
&= -\xi_{\perp}\left[(\nu_t\mathbf{a} +\nu\mathbf{a}_t + \eta_t\mathbf{b} + \eta\mathbf{b}_t)\cdot\mathbf{a}\right]\\
&= -\xi_{\perp}\left[(\nu_t\mathbf{a} +\nu\theta_t\mathbf{b} + \eta_t\mathbf{b} - \eta\theta_t\mathbf{a})\cdot\mathbf{a}\right]\\
&= -\xi_{\perp}(\nu\theta_t + \eta_t)\\
&= -\xi_{\perp}\nu\theta_t - \xi_{\perp}\eta_t.
\end{split}
\end{equation}
%%
By putting all together, we finally get three equations along $\mathbf{e}_3$, $\mathbf{a}$ and $\mathbf{b}$ respectively
\begin{align}
M_s + \nu H -\eta N 
&= \rho I \theta_{tt}\\
N_{ss} - 2\theta_sH_s - H\theta_{ss}- N\theta_s^2 
- \xi_{\parallel}\nu_t + \xi_{\parallel}\eta\theta_t
&= \rho S (\nu_{tt} - 2 \theta_t\eta_t - \eta\theta_{tt} - \nu\theta_t^2)\\
H_{ss} + 2\theta_sN_s + N\theta_{ss} - H\theta_s^2 
- \xi_{\perp}\nu\theta_t - \xi_{\perp}\eta_t
&= \rho S (\eta_{tt} + 2\theta_t\nu_t + \nu\theta_{tt} - \eta\theta_t^2).
\end{align}
The three constitutive equations 
\begin{align}
M &= C(\theta_s -\alpha) + \tilde{C}(\theta_s)_t\\
N &= A(\nu - 1) + \tilde{A}\nu_t\\
H &= B\eta + \tilde{B}\eta_t
\end{align}
should also be considered in the equations of motion. We finally get six equations in six variables, the three principal variables $\theta,\nu,\eta$ and the respective conjugate ones M, N, H.
\begin{align}
M_s + \nu H -\eta N - \rho I \theta_{tt}
&= 0\\
N_{ss} - 2\theta_sH_s - H\theta_{ss}- N\theta_s^2
- \xi_{\parallel}\nu_t + \xi_{\parallel}\eta\theta_t - \rho S \nu_{tt} + 2\rho S \theta_t\eta_t + \rho S \eta\theta_{tt} + \rho S \nu\theta_t^2
&= 0\\
H_{ss} + 2\theta_sN_s + N\theta_{ss} - H\theta_s^2  - \xi_{\perp}\nu\theta_t - \xi_{\perp}\eta_t
 - \rho S \eta_{tt} - 2\rho S \theta_t\nu_t -\rho S \nu\theta_{tt} + \rho S \eta\theta_t^2
&= 0\\
M - C(\theta_s -\alpha) - \tilde{C}(\theta_s)_t &= 0\\
N - A(\nu - 1) - \tilde{A}\nu_t &= 0\\
H - B\eta - \tilde{B}\eta_t &= 0
\end{align}



\\\\\\\\
( \emph{Per LUCA: Scrivo qua dei dubbi con le mie risposte & considerazioni che potrebbero risultare utili prima di trattare la parte dei test numerici. Spero di essermi espressa in maniera abbastanza chiara.
Il commento si può cancellare una volta letto.}
\\\\
Per quanto riguarda le equazioni costitutive, mi sono accertata che 
\[ \hat{M} = C(\theta_s -\alpha), \hat{N} = A(\nu - 1) \,\,e\,\, \hat{H} = B\eta \]
senza le derivate spaziali. Probabilmente all'inizio si è pensato di prendere 
\[ \hat{M} = C(\theta_s -\alpha)_s, \hat{N} = A(\nu - 1)_s, \hat{H} = B\eta _s, \] 
ricordando che in Jülicher compariva un 
\[ f_e = - \frac{\partial G}{\partial X} =(\eta n + \tau t)_s \]
ed $\eta$ risultava essere uguale a $\eta$ = - B$(\theta_s - \alpha)_s$. Però qua la situazione è leggermente diversa, perchè le cose sono molto più semplificate grazie all'equilibro imposto.
Infatti il valore di $\eta$ si ottiene considerando la derivata del funzionale energia; ecco perchè la derivata spaziale in più.
L'altra grande differenza è che nel caso Jülicher, $\eta$ rappresenta il valore della forza lungo la direzione normale $\mathbf{n}$; che tradotto alla nostra notazione sarebbe H. Quindi l'equazione in cui compare $\alpha$ in Jülicher è l'equazione della forza lungo la direzione normale (lungo H per noi).
\\\\
\emph{Quindi le due differenze che noto sono:}
\\\\
\emph{1. In Jülicher abbiamo $\eta$ = - B$(\theta_s - \alpha)_s$, che è l'equazione che mi da $\eta$, ovvero la componente della forza lungo la direzione normale ($"\mathbf{b}"$), ed è derivata rispetto ad s per il motivo che ho elencato sopra.}
\\\\
\emph{2. Noi abbiamo $\hat{M}$ = C($\theta$_s - $\alpha$), che è l'equazione lungo M e non è derivata rispetto ad s.}
\\\\
\emph{Le due equazioni hanno natura diversa.}
\\\\
Ciò nonnostante, se le cose sono state fatte bene, una volta che non si considerano più i termini inerziali nel problema vincolato (Section 5.2),  precisamente tolti $\rho S \mathbf{r}_{tt}$ e $\rho I \theta_{tt}$ nell'equazione (69), e nelle equazioni e matrici che seguono, si dovrebbe trovare un sistema che è equivalente al vecchio sistema che avevamo in Jülicher (Section 5.4). Giusto? )

\end{document}
 
  
  
  
  
  
  
  
  
  
  
  
  
  
  
  
  
  
  
  
  
  
  
  
  
  
  
  
  
  
  
  
  
  
  
  
  
  
  