\documentclass[12pt]{article}

\usepackage{pdfsync}
\usepackage{amsmath}
\usepackage{graphicx}
\usepackage{textcomp}
\usepackage[T1]{fontenc}

\graphicspath{{./figures/}}

%%%%%%%%%%%%%%%%%%%%%%%%%%%%%%%%%%%%%%%%%%%%%%%%%%%%%%%%%%%%
\renewcommand{\d}{\mathrm{d}}
\newcommand{\vv}[1]{\boldsymbol{#1}}
\newcommand{\cB}{B}
\newcommand{\cA}{A}
\newcommand{\cC}{C}



%%%%%%%%%%%%%%%%%%%%%%%%%%%%%%%%%%%%%%%%%%%%%%%%%%%%%%%%%%%%


\begin{document}
\section{Derivation of the Equations of Motion
}\label{Derivation of the Equations of Motion
}
\subsection{Some highlights of Hamilton's Principle}
\subsection{Derivation of the Equations of Motion}
$\bullet$\emph{The Lagrangian $\mathcal{L}$}. We define the kinetic energy density 
\[ \mathcal{T}(\mathbf{y}_t) = \frac{\rho S}{2} \mathbf{r}_t \cdot \mathbf{r}_t + \frac{\rho I}{2}\theta_t ^2,
\]
where $\rho$S is the linear mass density and $\rho$I the linear moment of inertia.
Under the action of forces, the elastic rod-like filament suffers deformations, which we assume to be planar. Whenever these deformations occur, it will be a contact force \textbf{n}(s,t) exerted by the material of [s,1] on that of [0,s). \textbf{n}(s,t) is assumed to lie to the (\textbf{e_1,e_2})-plane, so that it has the form
\[  \mathbf{n}(s,t)= N(s,t) \mathbf{a}(s,t) + H(s,t)\mathbf{b} (s,t).
\]
Let \mathbf{m}(s,t) be the contact couple exerted by the material of [s,1] on tha of [0,s). We assume that \mathbf{m} is perpendicular to the (\mathbf{e_1,e_2})-plane so that it has the form
\[ 
\mathbf{m}(s,t) = M(s,t)\mathbf{e}_3.
\]

We assume that the material of the rod is viscoelastic causing internal energy dissipation and having constitutive equations of the form
\[  N (s,t)=  \hat{N}(\nu (s,t),\eta (s,t)) + \tilde{A} \nu_t (s,t)
\]
\[  H (s,t)=  \hat{H}(\nu (s,t),\eta (s,t)) + \tilde{B} \eta_t (s,t)
\]
\[  M (s,t)=  \hat{M}(\nu (s,t),\eta (s,t), \mu(s,t)) + \tilde{C} \mu_t (s,t),
\]
where $\hat{N}$,$\hat{H}$,$\hat{M}$ are given functions on (0,$\infty$)x$\mathbb{R}$x$\mathbb{R}$ and where $\tilde{A}$, $\tilde{B}$, $\tilde{C}$ are positive constants.
\\
We model this by introducing the elastic potential density  as
\[ \mathcal{U}(\mathbf{y}) = \frac{\cA}{2} (\nu - \nu_0)^2
+ \frac{\cC}{2} (\eta - \eta_0 )^2
+ \frac{\cB}{2} (\mu -\alpha)^2,
\]
where $\alpha$ = $\alpha$(s,t) is \emph{spontaneous curvature}, $\nu$ = $\nu$(s,t) describes the stretch, $\eta$ = $\eta$(s,t) defines the shear strain and $\cA$, $\cB$, $\cC$ represent respectively the stretch stiffness, the bending stiffness and the shear stiffness. 
This formulation takes into account that we are dealing with a \emph{compressible} viscoelastic medium that has the ability to actively modify its spontaneous curvature.\\
Tha Lagrangian density $\mathcal{L}$ of the system will therefore read
\[ \mathcal{L}(\mathbf{y},\mathbf{y}_t)= \mathcal{T}(\mathbf{y}_t) + \mathcal{U}(\mathbf{y}).
\]
$\bullet$\emph{The Virtual Word $\delta$W}. Along with the elastic potential we introduce the internal energy dissipation density  and the dissipative force associated as 
\[ \mathcal{D}_{diss}(\mathbf{y}_t) = \frac{\tilde{A}}{2}\left|\mu_t\right|^2 + \frac{\tilde{B}}{2}\left|\eta_t\right|^2
+ \frac{\tilde{C}}{2}\left|\nu_t\right|^2,
\]

\[
\mathbf{Q}_{diss}= (\tilde{A}(\nu_t)_s,\tilde{B}(\eta_t)_s,\tilde{A}(\mu_t)_s).
\]
When immersed in a fluid, if we use \emph{resistive force theory},
then the external energy dissipation density and the force applied by fluid on the rod can be modeled as
\[ \mathcal{D}_{ext}(\mathbf{y}_t)= -(\xi_\parallel\left|\mathbf{r}_t\cdot\mathbf{\tau}\right|^2 + \xi_\perp \left|\mathbf{r}_t\cdot\mathbf{P}\right|^2),
\]

\[\mathbf{Q}_{ext} = -( \xi_\parallel (\mathbf{r}_t\otimes\mathbf{\tau})\mathbf{\tau} + \xi_\perp (\mathbf{r}_t\otimes\mathbf{P})\mathbf{P}),
\]
where $\xi_\parallel$ and $\xi_\perp$ are local friction coefficients fort tangential and normal motion, respectively.
We deal with the dissipative forces by means of virtual work terms
\[\delta W = \delta W_{diss} + \delta W_{ext} = (\mathbf{Q}_{diss}+\mathbf{Q}_{ext})\cdot\mathbf{y}.\]








\end{document}
  
  
  
  
  
  
  
  
  
  
  
  
  
  
  
  
  
  
  
  
  
  
  
  
  
  
  
  
  
  
  
  
  
  
  
  
  
  
  
  
  
  
  
  
  
  
  
  
  
  
  
  
  
  
  
  
  
  
  
  
  