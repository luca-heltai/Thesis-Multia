\documentclass[12pt]{article}

\usepackage{pdfsync}
\usepackage{amsmath}
\usepackage{graphicx}
\usepackage{textcomp}
\usepackage[T1]{fontenc}

\graphicspath{{./figures/}}

%%%%%%%%%%%%%%%%%%%%%%%%%%%%%%%%%%%%%%%%%%%%%%%%%%%%%%%%%%%%
\renewcommand{\d}{\mathrm{d}}
\newcommand{\vv}[1]{\boldsymbol{#1}}
\newcommand{\cB}{B}
\newcommand{\cA}{A}
\newcommand{\cC}{C}



%%%%%%%%%%%%%%%%%%%%%%%%%%%%%%%%%%%%%%%%%%%%%%%%%%%%%%%%%%%%


\begin{document}
\section{Introduction}
In this work we explore the planar, nonlinear dynamics of an extensional shearable rod by using the simple Cosserat model. The method of Cosserat dynamics for elastic structures is employed since it can accommodate to a good approximation the nonlinear behavior of complex elastic structures composed of materials with different constitutive properties, variable geometry and damping characteristics. With arbitrary boundary conditions, the Cosserat theory is used to formulate a set of governing partial differential equations of motion in terms of the displacements and angular variables, describing the planar, nonlinear dynamics of an extensional rod. Bending about two principal axes, extension, shear and torsion are considered, and care is taken into account for all the nonlinear terms in the resulting equations.We derive, through variational principles, the equations of motion for two special cases:...

\section{Background on the special Cosserat model}
We briefly summarize the basic aspects of the special Cosserat theory of rods using a notation adopted from Antman. Generally, a rod is a fiber–like elastic body, i.e. it is possible to specify a family of cross–sections which have small proportions compared to the length of the rod. This suggests a mathematical model of a rod in terms of a spacecurve corresponding to its centerline and a director frame which defines the orientation of the local
cross–section plane.
Throughout this work, we denote three–dimensional Euclidean space by ($\mathbf{E}$, ⟨ · , · ⟩) and choose a fixed right–handed orthonormal triple ($\mathbf{e}$_1, $\mathbf{e}${e}_2, $\mathbf{e}${e}_3) of basis vectors.
\subsection{Configuration}
Let $\mathbf{r}$(s,t) : [s_1,s_2] × [t_1,t_2] → $\mathbb{R}$^2 be a smooth space curve describing the centerline of the rod.
We work in a non-dimensionalized setting, that is, we assume that the length of the filament is equal to one.



\end{document}
  
  
  
  
  
  
  