\begin{document}
\subsubsection{Active flagella vs inextensible and unshearable Beck’s Column when neglecting the inertial terms}

\[
\begin{array}{cc}
\toprule
f(x)   & \text{Una primitiva} \\
\midrule
e^x    & e^x \\
\cos x & \sin x \\
\sin x & -\cos x \\
\bottomrule
\end{array}
\]

\begin{tabular}{lp{0.5\textwidth}}
\toprule
\textbf{Forza} & Una forza è una grandezza fisica che si manifesta
nell’interazione di due o più corpi materiali, che cambia lo stato
di quiete o di moto dei corpi stessi. \\
\midrule
\textbf{Momento polare} & Il momento polare di una forza rispetto a
una determinata origine è definito come il prodotto vettoriale tra
il vettore posizione (rispetto alla stessa origine) e la forza. \\
\bottomrule
\end{tabular}




\[
\begin{table} 
    \begin{tabular}{ c c c }
        0 & Active flagella & Inextensible and unshearable Beck’s Column when neglecting the inertial terms \\ 
        Forces & f_f = = -\xi_{\parallel}(\mathbf{r}_t\cdot\mathbf{a})\mathbf{a}- \xi_{\perp}(\mathbf{r}_t\cdot\mathbf{b})\mathbf{b} \\ f_e = -\frac{\partial \mathcal{G}}{\partial \mathbf{r}} = (H \mathbf{b} + N \mathbf{a})_s 
         &  \mathbf{F}_{extart} = -\xi_{\parallel}(\mathbf{r}_t\cdot\mathbf{a})\mathbf{a}- \xi_{\perp}(\mathbf{r}_t\cdot\mathbf{b})\mathbf{b}\\ \mathbf{\tilde{T}}_s= (\tilde{N}\mathbf{a} + \tilde{H}\mathbf{b})_s
         \\
       First equation of motion  & From equilibrium we have f_e + f_f = 0: \\  
       (H \mathbf{b} + N \mathbf{a})_s
       & (N\mathbf{a} + H\mathbf{b})_s -\xi_{\parallel}(\mathbf{r}_t\cdot\mathbf{a})\mathbf{a}- \xi_{\perp}(\mathbf{r}_t\cdot\mathbf{b})\mathbf{b} = 0
            
        & (\tilde{N}\mathbf{a} + \tilde{H}\mathbf{b})_s -\xi_{\parallel}(\mathbf{r}_t\cdot\mathbf{a})\mathbf{a}- \xi_{\perp}(\mathbf{r}_t\cdot\mathbf{b})\mathbf{b} = 0   \\ 
       Remaining equations of motion  & H =−(B(\kappa−\alpha))_s &  
       M - C(\theta_s -\alpha)  = 0\\
\tilde{N}(\nu - 1) = 0\\
\tilde{H}\eta = 0 \\
where \mathbf{r}_s = \nu\mathbf{a}+\eta\mathbf{b} 
                \\ 
         0& 0 &0 \\ 
    \end{tabular} 
\end{table}
\]
\end{document}