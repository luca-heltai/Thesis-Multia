\documentclass[12pt]{article}

\usepackage{pdfsync}
\usepackage{amsmath}
\usepackage{graphicx}

\graphicspath{{./figures/}}

%%%%%%%%%%%%%%%%%%%%%%%%%%%%%%%%%%%%%%%%%%%%%%%%%%%%%%%%%%%%
\renewcommand{\d}{\mathrm{d}}
\newcommand{\vv}[1]{\boldsymbol{#1}}
\newcommand{\cB}{B}
\newcommand{\cA}{A}
\newcommand{\cC}{C}


%%%%%%%%%%%%%%%%%%%%%%%%%%%%%%%%%%%%%%%%%%%%%%%%%%%%%%%%%%%%


\begin{document}
\section{Derivation of the Equations of Motion
}\label{Derivation of the Equations of Motion
}
\subsection{Hamilton's Principle}
\subsection{Derivation of the Equations of Motion}
Under the action of forces, the elastic rod-like filament suffers deformations, which we assume to be planar. Whenever these deformations occur, it will be a contact force $n$(s,t) exerted by the material of [s,1] on that of [0,s). $n$(s,t) is assumed to lie to the ($e_1$,$e_2$)-plane, so that it has the form
\[  $n$(s,t)=$N$(s,t)$a$(s,t) + $H$(s,t)$b$(s,t)
\]
We model this by introducing the elastic potential density as
\[ \mathcal{U}(\textbf {y}) = \frac{\cA}{2} (\nu - \nu_0)^2
+\frac{\cC}{2} (\eta - \eta_0 )^2
+\frac{\cB}{2} (\mu -\alpha)^2 
\]
where
\end{document}
  
  
  
  