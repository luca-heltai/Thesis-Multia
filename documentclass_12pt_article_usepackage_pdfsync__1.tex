\documentclass[12pt]{article}

\usepackage{pdfsync}
\usepackage{amsmath}
\usepackage{graphicx}
\usepackage{textcomp}
\usepackage[T1]{fontenc}

\graphicspath{{./figures/}}

%%%%%%%%%%%%%%%%%%%%%%%%%%%%%%%%%%%%%%%%%%%%%%%%%%%%%%%%%%%%
\renewcommand{\d}{\mathrm{d}}
\newcommand{\vv}[1]{\boldsymbol{#1}}
\newcommand{\cB}{B}
\newcommand{\cA}{A}
\newcommand{\cC}{C}



%%%%%%%%%%%%%%%%%%%%%%%%%%%%%%%%%%%%%%%%%%%%%%%%%%%%%%%%%%%%


\begin{document}
\section{Derivation of the Equations of Motion
}\label{Derivation of the Equations of Motion
}
\subsection{Some highlights of Hamilton's Principle}
..work in progress.. :D
\subsection{Derivation of the Equations of Motion}
$\bullet$\emph{The Lagrangian $\mathcal{L}$}. We define the kinetic energy density 
\[ \mathcal{T}(\mathbf{y}_t) = \frac{\rho S}{2} \mathbf{r}_t \cdot \mathbf{r}_t + \frac{\rho I}{2}\theta_t ^2,
\]
where $\rho$S is the linear mass density and $\rho$I the linear moment of inertia.
Under the action of forces, the elastic rod-like filament suffers deformations, which we assume to be planar. Whenever these deformations occur, there is a contact force
$\mathbf{n}$(s,t) exerted by the material of [s,1] on that of [0,s). $\mathbf{n}$(s,t) is assumed to lie to the ($\mathbf{e}_1$,$\mathbf{e}$_2)-plane, so that it has the form
\[  \mathbf{n}(s,t)= N(s,t) \mathbf{a}(s,t) + H(s,t)\mathbf{b} (s,t).
\]
Let $\mathbf{m}$(s,t) be the contact couple exerted by the material of [s,1] on that of [0,s). We assume that $\mathbf{m}$ is perpendicular to the ($\mathbf{e}_1$,$\mathbf{e}_2$)-plane so that it has the form
\[ 
\mathbf{m}(s,t) = M(s,t)\mathbf{e}_3.
\]

We assume that the material of the rod is viscoelastic causing internal energy dissipation and having constitutive equations of the form
\[  N (s,t)=  \hat{N}(\nu (s,t),\eta (s,t)) + \tilde{A} \nu_t (s,t)
\]
\[  H (s,t)=  \hat{H}(\nu (s,t),\eta (s,t)) + \tilde{B} \eta_t (s,t)
\]
\[  M (s,t)=  \hat{M}(\nu (s,t),\eta (s,t), \mu(s,t)) + \tilde{C} \mu_t (s,t),
\]
where $\hat{N}$,$\hat{H}$,$\hat{M}$ are given functions on (0,$\infty$)x$\mathbb{R}$x$\mathbb{R}$ and where $\tilde{A}$, $\tilde{B}$, $\tilde{C}$ are positive constants.
\\
We model this by introducing the elastic potential density  as
\[ \mathcal{U}(\mathbf{y}) = \frac{\cA}{2} (\nu - \nu_0)^2
+ \frac{\cC}{2} (\eta - \eta_0 )^2
+ \frac{\cB}{2} (\mu -\alpha)^2,
\]
where $\alpha$ = $\alpha$(s,t) is \emph{spontaneous curvature}, $\nu$ = $\nu$(s,t) describes the stretch, $\eta$ = $\eta$(s,t) defines the shear strain and $\cA$, $\cB$, $\cC$ represent respectively the stretch stiffness, the bending stiffness and the shear stiffness. We assume that $\nu_0\equiv1$ and that $\eta_0\equiv0$.
This formulation takes into account that we are dealing with a \emph{compressible} viscoelastic medium that has the ability to actively modify its spontaneous curvature.\\
Tha Lagrangian density $\mathcal{L}$ of the system will therefore read
\[ \mathcal{L}(\mathbf{y},\mathbf{y}_t)= \mathcal{T}(\mathbf{y}_t) - \mathcal{U}(\mathbf{y}).
\]
$\bullet$\emph{The Virtual Work $\delta$W}. Along with the elastic potential we introduce the internal energy dissipation density  and the dissipative force associated as 
\[ \mathcal{D}_{diss}(\mathbf{y}_t) = \frac{\tilde{A}}{2}\left|\mu_t\right|^2 + \frac{\tilde{B}}{2}\left|\eta_t\right|^2
+ \frac{\tilde{C}}{2}\left|\nu_t\right|^2,
\]

\[
\mathbf{Q}_{diss}= (\tilde{A}(\nu_t)_s,\tilde{B}(\eta_t)_s,\tilde{A}(\mu_t)_s).
\]
When immersed in a fluid, if we use \emph{resistive force theory},
then the external energy dissipation density and the force applied by fluid on the rod can be modeled as
\[ \mathcal{D}_{ext}(\mathbf{y}_t)= -(\xi_\parallel\left|\mathbf{r}_t\cdot\mathbf{\tau}\right|^2 + \xi_\perp \left|\mathbf{r}_t\cdot\mathbf{P}\right|^2),
\]

\[\mathbf{Q}_{ext} = -( \xi_\parallel (\mathbf{r}_t\otimes\mathbf{\tau})\mathbf{\tau} + \xi_\perp (\mathbf{r}_t\otimes\mathbf{P})\mathbf{P}),
\]
where $\xi_\parallel$ and $\xi_\perp$ are local friction coefficients for tangential and normal motion, respectively.
We deal with the dissipative forces by means of virtual work terms
\[\delta W = \delta W_{diss} + \delta W_{ext} = (\mathbf{Q}_{diss}+\mathbf{Q}_{ext})\cdot\mathbf{y}.
\]
According to the Hamilton's Principle, a solution $\mathbf{y}$ must satisfy
\[\int_{t_1}^{t_2} \int_{0}^{1} \delta \mathcal{L}+\delta W\, ds \, dt = 0
\]
for every perturbation $\delta \mathbf{y}$ defined on $[0,1]x[t_1,t_2]$ and vanishing at its boundary. We consider 

\begin{equation}
\begin{split}
 \delta\mathcal{T}& = \frac{\partial \mathcal{T}(\mathbf{y}_t + \epsilon (\delta \mathbf{y})_t )}{\partial{\epsilon}} \arrowvert_{\epsilon=0}\\
                  & = \rho S\mathbf{r}_t (\delta\mathbf{r})_t + \rho I \theta_t (\delta \theta)_t\\
\end{split}
\end{equation}

and take the integral of the total kinetic energy perturbation with respect to time from t_1 to t_2
\[\int_{t_1}^{t_2} \int_{0}^{1} \delta \mathcal{T}\, ds \, dt = \int_{t_1}^{t_2} \int_{0}^{1} \rho S\mathbf{r}_t (\delta\mathbf{r})_t + \rho I \theta_t (\delta \theta)_t\, ds \, dt.
\]
Integrating the expression on the right by parts with respect to time and recalling that $\delta\mathbf{y}$ vanishes at t_1 and t_2  yields
\[\int_{t_1}^{t_2}\int_{0}^{1}\delta \mathcal{T}\,ds\,dt = -\int_{t_1}^{t_2}\int_{0}^{1}(\rho S\mathbf{r}_{tt}, \rho I\theta_{tt})\cdot(\delta\mathbf{r},\delta\theta)\,ds\,dt.
\]
The same argument yields
\begin{equation}
\begin{split}
 \delta\mathcal{U}& = \frac{\partial \mathcal{U}(\mathbf{y} + \epsilon\delta\mathbf{y})}{\partial{\epsilon}} \arrowvert_{\epsilon=0}\\
                  & = \cA(\nu-\nu_{0})\delta\nu+\cC(\eta-\eta_0)\delta\eta+\cB(\mu-\mu_0)\delta\mu.\\
\end{split}
\end{equation}
We take the integral of the total potential perturbation with respect to time from t_1 to t_2
\begin{equation}
\begin{split}
 \delta\mathcal{U}& = \int_{t_1}^{t_2}\int_{0}^{1}\delta\mathcal{U}\,ds\,dt \\
                  & = \int_{t_1}^{t_2} \int_{0}^{1} 
(\cA(\nu-\nu_{0}),\cC(\eta-\eta_0),\cB(\mu-\mu_0))\cdot(\delta\nu,\delta\eta,\delta\mu)\,ds\,dt\\
                  & = \int_{t_1}^{t_2} \int_{0}^{1} 
(\cA(\nu-\nu_{0}),\cC(\eta-\eta_0),\cB(\mu-\mu_0))\cdot(\delta \mathcal{r}_s,\delta\theta_s)\,ds\,dt.\\
\end{split}
\end{equation}
Integrating the expression on the right by parts with respect to s and recalling that $\delta\mathbf{y}$ vanishes at \{\ 0,1\}\ gives
\[\int_{t_1}^{t_2}\int_{0}^{1}\delta\mathcal{U}\,ds\,dt =
-\int_{t_1}^{t_2} \int_{0}^{1} 
(\cA(\nu-\nu_{0})_s,\cC(\eta-\eta_0)_s,\cB(\mu-\mu_0)_s)\cdot(\delta\mathbf{r},\delta\theta)\,ds\,dt.
\]

Finally,
\begin{equation}
\begin{split}
 \int_{t_1}^{t_2}\int_{0}^{1}\delta W\,ds\,dt &=
\int_{t_1}^{t_2} \int_{0}^{1} \mathcal{Q}_{diss}
\,ds\,dt +
\int_{t_1}^{t_2} \int_{0}^{1} \mathcal{Q}_{ext}
\,ds\,dt\\
                                              & = \int_{t_1}^{t_2} \int_{0}^{1} (\tilde{A}(\nu_t)_s,\tilde{B}(\eta_t)_s,\tilde{A}(\mu_t)_s)\cdot (\delta\mathbf{r},\delta\theta)
\,ds\,dt - \int_{t_1}^{t_2} \int_{0}^{1}  (\xi_\parallel (\mathbf{r}_t\cdot\mathbf{\tau})\mathbf{\tau} + \xi_\perp (\mathbf{r}_t\cdot\mathbf{P})\mathbf{P})\cdot \delta\mathbf{r}\,ds\,dt
\end{split}
\end{equation}
After putting all together we have
\begin{equation}
\begin{split}
 \int_{t_1}^{t_2} \int_{0}^{1} \delta \mathcal{L}+\delta W\, ds\,dt= \\ 
         &= \int_{t_1}^{t_2} \int_{0}^{1}
         \lgroup 
         -\ro S \mathbf{r}_{tt}\\
         +(\cA(\nu -\nu_0)_s,\cC(\eta -\eta_0)_s)\\
         + (\tilde_{A}(\nu_t)_s,\tilde_{B}(\eta_t)_s)\\
         - (\xi_\parallel (\mathbf{r}_t\cdot\mathbf{\tau})\mathbf{\tau} + \xi_\perp (\mathbf{r}_t\cdot\mathbf{P})\mathbf{P})
         \rgroup\\
         \cdot\delta\mathbf{r}\,ds\,dt\\
         + \int_{t_1}^{t_2} \int_{0}^{1}
         \lgroup-\ro I \theta_{tt}\\
         + \cB(\mu-\alpha)_s\\ 
         + \tilde_{C}(\mu_t)_s\rgroup 
         \cdot\delta\mathbf{r}\,ds\,dt
\end{split}
\end{equation}
Since the Hamilton's Principle must hold, it forces the coefficients multiplying $\delta\mathbf{r}$ and $\delta\theta$ to vanish. The equations of motion then read
\[
(\cA(\nu-\nu_0)_s +\tilde{A}(\nu_t)_s, \cC(\eta-\eta_0)_s +\tilde{B}(\eta_t)_s)- (\xi_\parallel (\mathbf{r}_t\cdot\mathbf{\tau})\mathbf{\tau} + \xi_\perp (\mathbf{r}_t\cdot\mathbf{P})\mathbf{P}) & = & \rho S \mathbf{r}_{tt} \\
\cB(\mu-\alpha)_s + \tilde_{C}(\mu_t)_s  = \rho I \theta_{tt}
\]
i.e.
\[
\mathbf{n}_s - (\xi_\parallel (\mathbf{r}_t\cdot\mathbf{\tau})\mathbf{\tau} + \xi_\perp (\mathbf{r}_t\cdot\mathbf{P})\mathbf{P}) = \rho S \mathbf{r}_{tt}\\
M_s=\rho I \theta_{tt}
\]

\subsection{Constraints}
We treat now the case of a incompressible viscoelastic rod,
we write the Hamilton's Principle and derive the equations of motion for this case.
We begin by imposing the constraints that the rod is inextensible and unsharable:
\begin{eqnarray*}
\nu(s,t)=1 \\ 
\eta(s,t)=0.
\end{eqnarray*}
Let $\bar{N}=\bar{N}$(s,t) and $\bar{H}=\bar{H}$(s,t) denote the \emph{Lagrange multipliers} representing the reactive internal tractions (axial tension and shear force, respectively) enforcing the constraints.
We give them the same letters as the components of contact force, because morally, from a physic point of view they have the same meaning.
We define
\[
\mathcal{C}(s,t) = N(\nu-1) + H\eta
\]
and consider
\[\int_{t_1}^{t_2} \int_{0}^{1} \delta \mathcal{C}\, ds \, dt = \int_{t_1}^{t_2} \int_{0}^{1} ((\bar{N}\mathbf{a})_s + (\bar{H}\mathbf{b})_s), \nu\bar{H}-\eta\bar{N})\cdot(\delta \mathbf{r},\delta\theta)
\,ds \, dt.
\]
It can be trivially shown by replacing the equations of the constraints that the elastic potential density becomes
\[ \mathcal{U}^{constr}(\mathbf{y}) = \frac{\cB}{2} (\mu -\alpha)^2,
\]
the internal dissipative energy density becomes
\[ \mathcal{D}_{diss}^{constr}(\mathbf{y}_t) = \frac{\tilde{C}}{2}\left|\mu_t\right|^2,
\]
while the kinetic energy density and the external energy dissipation density remain the same.
The Hamilton's Priciple then states
\[\int_{t_1}^{t_2} \int_{0}^{1} \delta (\mathcal{T}-\mathcal{U^{constr}}) + \delta \mathcal{C} ds \, dt
+ \int_{t_1}^{t_2} \int_{0}^{1} (\mathbf{Q}_{diss}^{constr} +
\mathbf{Q}_{ext})\cdot (\delta \mathbf{r},\delta\theta)
\, ds \, dt = 0
\]
for every perturbation $\delta \mathbf{y}$ defined on $[0,1]x[t_1,t_2]$ and vanishing at its boundary. By replacing the value of each term, we find
\[\int_{t_1}^{t_2} \int_{0}^{1} \delta (\mathcal{T}-\mathcal{U^{constr}}) + \delta \mathcal{C} ds \, dt
+ \int_{t_1}^{t_2} \int_{0}^{1} (\mathbf{Q}_{diss}^{constr} +
\mathbf{Q}_{ext})\cdot (\delta \mathbf{r},\delta\theta)
\, ds \, dt =
\int_{t_1}^{t_2} \int_{0}^{1}
         -(\ro S \mathbf{r}_{tt}
         +(\cA(\nu -\nu_0)_s,\cC(\eta -\eta_0)_s)\\
         + (\tilde_{A}(\nu_t)_s,\tilde_{B}(\eta_t)_s)
         - (\xi_\parallel (\mathbf{r}_t\cdot\mathbf{\tau})\mathbf{\tau} + \xi_\perp (\mathbf{r}_t\cdot\mathbf{P})\mathbf{P} + (\bar{N}\mathbf{a})_s + (\bar{H}\mathbf{b})_s) )
         \cdot\delta\mathbf{r}\,ds\,dt
         + \int_{t_1}^{t_2} \int_{0}^{1}
         (-\ro I \theta_{tt}
         + \cB(\mu-\alpha)_s 
         + \tilde_{C}(\mu_t)_s)
         +\nu\bar{N}-\eta\bar{H})
         \cdot\delta\theta \,ds\,dt
\]
and the equations of motion become
\[ (\bar{N}\mathbf{a} + (\bar{H}\mathbf{b})_s - (\xi_\parallel (\mathbf{r}_t\cdot\mathbf{\tau})\mathbf{\tau} + \xi_\perp (\mathbf{r}_t\cdot\mathbf{P})\mathbf{P}) = \rho S \mathbf{r}_{tt}\\
M_s + \mathbf{e}_3\cdot(\mathbf{r}_s\cdot(\bar{N}\mathbf{a} + \bar{H}\mathbf{b}))=\rho I \theta_{tt}
\]













\end{document}
  
  
  
  
  
  
  
  
  
  
  
  
  
  
  
  
  
  
  
  
  
  
  
  
  
  
  
  
  
  
  
  
  
  
  
  
  
  
  
  
  
  
  
  
  
  
  
  
  
  
  
  
  
  
  
  
  
  
  
  
  
  
  
  
  
  
  
  
  
  
  
  
  
  
  
  
  
  
  
  
  
  
  
  
  
  
  
  
  
  
  
  
  
  
  
  
  
  
  
  
  
  
  
  
  
  
  
  
  
  
  
  
  
  
  
  
  
  
  
  
  
  
  
  
  
  
  
  
  
  
  
  
  
  